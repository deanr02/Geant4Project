\documentclass[%
 reprint,
%superscriptaddress,
%groupedaddress,
%unsortedaddress,
%runinaddress,
%frontmatterverbose, 
%preprint,
%preprintnumbers,
%nofootinbib,
%nobibnotes,
%bibnotes,
 amsmath,amssymb,
 aps,
%pra,
%prb,
%rmp,
%prstab,
%prstper,
%float,
]{revtex4-2}
\usepackage{url}
\usepackage{float}
\usepackage{graphicx}% Include figure files
\graphicspath{{../images/}}
\usepackage{dcolumn}% Align table columns on decimal point
\usepackage{bm}% bold math
%\usepackage{hyperref}% add hypertext capabilities
%\usepackage[mathlines]{lineno}% Enable numbering of text and display math
%\linenumbers\relax % Commence numbering lines

%\usepackage[showframe,%Uncomment any one of the following lines to test 
%%scale=0.7, marginratio={1:1, 2:3}, ignoreall,% default settings
%%text={7in,10in},centering,
%%margin=1.5in,
%%total={6.5in,8.75in}, top=1.2in, left=0.9in, includefoot,
%%height=10in,a5paper,hmargin={3cm,0.8in},
%]{geometry}

\begin{document}

\preprint{APS/123-QED}

\title{Simulation of a Liquid Argon Electromagnetic Calorimeter in Geant4}% Force line breaks with \\

\author{Dean Reiter}
\affiliation{%
 Cornell University\\
}%

\date{\today}% It is always \today, today,
             %  but any date may be explicitly specified

\begin{abstract}


\end{abstract}


\maketitle



\section{Introduction}

What is Geant4 ?
-summary
-how it works, what it models
-when, how, who its used

Scope and goals
-goal to learn geant4, motivation
-electromagnetic calorimeter 
    - what is a calorimeter, why
    - electromagnetic vs hadronic
    - sampling vs homogenous
    - issues and engineering: efficiency, resolution, sampling fraction
-scope: modeling basic calorimeter
    - example models

\section{Simulation Setup}

- Detector model
    - based on example b4d -- modified DetectorConstruction, RunAction, EventAction
    - geometry: layers, absorber, gap
        - change: geom calculated from layer thickness and material ratio
        - large size, so energy can only leak by reflecting out incident surface
    - scoring volumes, energy deposit
        - change: created scorers for each layer, removed track scorers
- event 
    - what is an event
    - particlegun - electron, variable energy
    - steps, tracks, scoring
    - physics list
    - end of event
        -change: 2d histograms, profiles, energy deposition, removed track info
- run 
    - what is a run
    - run initialization
    - analysis manager, end of run
        - file save
- execution and computation
    - macro file to run 1000 events for various beam E
    - powershell script to automate runs for various geometries

\begin{figure}[H]
\centering
\includegraphics[width=0.75\columnwidth]{shower.png}% Here is how to import EPS art
\caption{\label{fig:epsart} Visualization of a simulated particle shower produced by a 1 GeV electron in a sampling calorimeter. Cross-sectional view is shown, with the electron incident from the right. The liquid argon and lead layers (white wire-frame) are 8 cm and 2 cm thick, respectively. Tracks are shown for particles with negative charge (red), positive charge (blue), and neutral charge (green). }
\end{figure}


\section{Data and Analysis}









\section{Conclusion}



\begin{acknowledgments}

[Acknowledgements]

\end{acknowledgments}


\appendix


\bibliography{main}% Produces the bibliography via BibTeX.

\end{document}
%
% ****** End of file apssamp.tex ******
